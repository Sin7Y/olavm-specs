\subsection{Trace Generator}\label{subsec: : trace-generator}
当OlaVM在执行编译好的指令时,它会在每执行一条指令时,将对应的内存状态和寄存器状态保存在一张trace表中。
\subsubsection{Main Trace}\label{subsubsec: olavm-main-trace}
trac主表的列分为5部分context, instruction, register selectors, opcode selectors, builtin selectors。


\textbf{context columns}

主trace的context相关列一共12列。其中VM的时钟clk和取指令地址pc各占一列。flag状态标志占一列。寄存器状态从$\texttt{r}_0 - \texttt{r}_{8}$一共9列。
注意registers的9列存的是执行前的状态。
\begin{itemize}
    \item \verb|clk|:VM的clock的缩写。操作内存时的VM时钟。在addr相同的情况下,clk是单调增的。
    \item \verb|pc|: VM操作内存时的指令地址。
    \item \verb|flag|: VM操作内存时的指令地址。
    \item \verb|registers|: 寄存器状态从$\texttt{r}_0 - \texttt{r}_{8}$一共9列。
\end{itemize}

context相关列的定义见下表。
\begin{table}[!ht]
    \centering \resizebox{0.6\linewidth}{!}{
        \begin{tabular}{*{6}{|c}|}
            \hline
            clk & pc  & flag & r0 & $\cdots$ & r8  \\ \hline
        \end{tabular}}
    \caption{OlaVM的main trace表context相关列结构}
    \label{table: olavm-trace-table-context}
\end{table}

\textbf{instruction columns}

主trace的instruction相关列一共4列。定义如下:
\begin{itemize}
    \item \verb|instruction|:指令的编码, 具体定义见表 \ref{table: processor_instruction_decode}。
    \item \verb|op1_imm|: 指令是否带立即数。1为带,0为不带。
    \item \verb|opcode|:  指令的opcode\_sel指示。具体见下边列表 \ref{table: olavm-trace-table-opcode-value-list}。
    \item \verb|immediate_value|: 如果op1\_imm为1,则存放指令带的立即数,否则为0。
\end{itemize}

\begin{table}[!ht]
    \centering \resizebox{\linewidth}{!}{
        \begin{tabular}{*{10}{|c}|}
            \hline
            ADD & MUL & EQ & ASSERT & MOV  & JMP & CJMP & CALL & RET & MLOAD  \\ \hline
            $2^{34}$ & $2^{33}$ & $2^{32}$ & $2^{31}$ & $2^{30}$ & $2^{29}$ & $2^{28}$ & $2^{27}$ & $2^{26}$ & $2^{25}$ \\ \hline
            MSTORE & END & RANGE\_CHECK & AND & OR & XOR & NOT & NEQ & GTE & \\ \hline
            $2^{24}$ & $2^{23}$ & $2^{22}$ & $2^{21}$ & $2^{20}$ & $2^{19}$ & $2^{18}$ & $2^{17}$ & $2^{16}$ & \\ \hline
        \end{tabular}}
    \caption{OlaVM的main trace表opcode value列表}
    \label{table: olavm-trace-table-opcode-value-list}
\end{table}



\textbf{register selector columns}

主trace的register selector相关列一共31列。定义如下:
\begin{itemize}
    \item \verb|op0|:操作数0的值。
    \item \verb|op1|:操作数1的值。
    \item \verb|dst|:目的寄存器的值。
    \item \verb|aux0|:辅助列,个别指令会用到。
    \item \verb|op0 regster selectors|:指示op0使用通用寄存器op0\_reg(参与op0列与指示的context里op0\_reg的copy约束)。
    \item \verb|op1 regster selectors|:指示op1使用通用寄存器op1\_reg(参与op1列与指示的context里op1\_reg的copy约束)。
    \item \verb|destination regster selectors|:指示dst使用通用寄存器dst\_reg(参与dst列与指示的context里dst\_reg的copy约束)。
\end{itemize}

register selector相关列的定义见下表。
\begin{table}[!ht]
    \centering \resizebox{\linewidth}{!}{
        \begin{tabular}{*{13}{|c}|}
            \hline
            op0 & op1 & dst & aux0 & sel\_op0\_r0 & $\cdots$ & sel\_op0\_r8 & sel\_op1\_r0 & $\cdots$ & sel\_op1\_r8 & sel\_dst\_r0 & $\cdots$ & sel\_dst\_r8 \\ \hline
        \end{tabular}}
    \caption{OlaVM的main trace表register selector相关列结构}
    \label{table: olavm-trace-table-register-selector}
\end{table}

\textbf{opcode selector columns}

主trace的opcode selector相关列一共12列。指示本行指令使用的opcode。相关列的定义见下表。
\begin{table}[!ht]
    \centering \resizebox{\linewidth}{!}{
        \begin{tabular}{*{12}{|c}|}
            \hline
            sel\_add & sel\_mul & sel\_eq & sel\_assert & sel\_mov & sel\_jmp & sel\_cjmp & sel\_call & sel\_ret & sel\_mload & sel\_mstore & sel\_end  \\ \hline
        \end{tabular}}
    \caption{OlaVM的main trace表opcode selector相关列结构}
    \label{table: olavm-trace-table-opcode-selector}
\end{table}

\textbf{builtin selector columns}

主trace的builtin selector相关列一共7列。指示本行指令使用的builtin。相关列的定义见下表。
\begin{table}[!ht]
    \centering \resizebox{0.6\linewidth}{!}{
        \begin{tabular}{*{7}{|c}|}
            \hline
            sel\_range\_check & sel\_and & sel\_or & sel\_xor & sel\_not & sel\_neq & sel\_gte  \\ \hline
        \end{tabular}}
    \caption{OlaVM的main trace表builtin selector相关列结构}
    \label{table: olavm-trace-table-builtin-selector}
\end{table}



\subsubsection{Memory Trace}\label{subsubsec: olavm-trace-memory}
OlaVM的内存trace由五列组成
\begin{itemize}
    \item \verb|addr|: 内存地址address的缩写。在表中是单调增的。
    \item \verb|clk|:VM的clock的缩写。操作内存时的VM时钟。在addr相同的情况下,clk是单调增的。
    \item \verb|rw|: VM操作内存时的操作状态,目前状态有两种:0代表读和1代表写。
    \item \verb|value|: 内存当前存储的值。
    \item \verb|prophet|: 如果是prophet操作内存,该字段为1,指令操作内存,该字段为0。
\end{itemize}

格式见下表 \ref{table: olavm-trace-memory-table}。

\begin{table}[!ht]
    \centering \resizebox{0.4\linewidth}{!}{
        \begin{tabular}{*{5}{|c}|}
            \hline
            addr & clk & rw & value & prophet    \\ \hline
        \end{tabular}}
    \caption{OlaVM的trace表结构}
    \label{table: olavm-trace-memory-table}
\end{table}

下表 \ref{table: olavm-trace-memory-table-example} 是一个内存子trace table的例子。按照后续约束规则,首先按照addr升序排列,同一个addr按照clk升序排列,如果该addr在未进行写操作之前进行读操作,读取值为默认0。
该表每行的addr和clk两个字段的组合可以唯一标识该行,注意内存地址102在第19个clk时,被prophet进行了写操作:

\begin{table}[!ht]
    \centering \resizebox{0.4\linewidth}{!}{
        \begin{tabular}{*{6}{|c}|}
            \hline
            addr & clk & pc & rw & value & prophet   \\ \hline
            100 & 4 & 3 & 1 & 0x25 & 0   \\ \hline
            100 & 7 & 5 & 0 & 0x25 & 0    \\ \hline
            100 & 11 & 7 & 0 & 0x25 & 0    \\ \hline
            100 & 15 & 3 & 1 & 0x74 & 0    \\ \hline
            100 & 17 & 6 & 0 & 0x74 & 0    \\ \hline
            102 & 8 & 6 & 1 & 0x11 & 0    \\ \hline
            102 & 13 & 9 & 0 & 0x11 & 0    \\ \hline
            102 & 19 & 19 & 1 & 0x53 & 1    \\ \hline
            102 & 20 & 20 & 0 & 0x53 & 0    \\ \hline
        \end{tabular}}
    \caption{OlaVM的memory trace表例子}
    \label{table: olavm-trace-memory-table-example}
\end{table}

\subsubsection{Range Check trace}\label{subsubsec: olavm-trace-range-check}
OlaVM的range check trace由5列组成
\begin{itemize}
    \item \verb|input value|: 要range check的值。
    \item \verb|limb0|: check value的第0-7位。
    \item \verb|limb1|: check value的第8-15位。
    \item \verb|limb2|: check value的第16-23位。
    \item \verb|limb3|: check value的第24-31位。
\end{itemize}

格式见下表 \ref{table: olavm-trace-range-check-table}。

\begin{table}[!ht]
    \centering \resizebox{0.4\linewidth}{!}{
        \begin{tabular}{*{5}{|c}|}
            \hline
            input value & limb0 & limb1 & limb2 & limb3     \\ \hline
        \end{tabular}}
    \caption{OlaVM的range check的trace表结构}
    \label{table: olavm-trace-range-check-table}
\end{table}

\subsubsection{Bitwise trace}\label{subsubsec: olavm-bitwise-trace}
OlaVM的bitwise trace由22列组成。同时bitwise操作的两个输入数会放入builtin的range check表。
\begin{itemize}
    \item \verb|clk|: 该bitwise操作对应在主trace的时钟周期。
    \item \verb|sel|: 选择bitwise的操作类型,0, 1, 2分别对应AND, OR, XOR。
    \item \verb|op1_value|: bitwise操作的第一个输入值。
    \item \verb|op2_value|: bitwise操作的第二个输入值。
    \item \verb|res_value|: bitwise操作的结果值。
    \item \verb|target_value|: op1, op2, res的limbs的组合结果,用于进行lookup。traget=op1 $\cdot 2^{16}$+op2 $\cdot 2^8$+res
    \item \verb|limb op1_0|: op1 value的第0-7位。
    \item \verb|limb op1_1|: op1 value的第8-15位。
    \item \verb|limb op1_2|: op1 value的第16-23位。
    \item \verb|limb op1_3|: op1 value的第24-31位。
    \item \verb|limb op2_0|: op2 value的第0-7位。
    \item \verb|limb op2_1|: op2 value的第8-15位。
    \item \verb|limb op2_2|: op2 value的第16-23位。
    \item \verb|limb op2_3|: op2 value的第24-31位。
    \item \verb|limb res_0|: res value的第0-7位。
    \item \verb|limb res_1|: res value的第8-15位。
    \item \verb|limb res_2|: res value的第16-23位。
    \item \verb|limb res_3|: res value的第24-31位。
    \item \verb|limb target_0|: target value的第0-7位。
    \item \verb|limb target_1|: target value的第8-15位。
    \item \verb|limb target_2|: target value的第16-23位。
    \item \verb|limb target_3|: target value的第24-31位。
\end{itemize}

格式见下表 \ref{table: olavm-trace-bitwise-table}。

\begin{table}[!ht]
    \centering \resizebox{\linewidth}{!}{
        \begin{tabular}{*{12}{|c}|}
            \hline
            clk & sel & op1\_value & op2\_value & res\_value & target\_value & op1\_0 & op1\_1 & op1\_2 & op1\_3 & op2\_0 & op2\_1 \\ \hline
            op2\_2 & op2\_3 & res\_0 & res\_1 & res\_2 & res\_3 & target\_0 & target\_1 & target\_2 & target\_3  & & \\ \hline
        \end{tabular}}
    \caption{OlaVM的bitwise的trace表结构}
    \label{table: olavm-trace-bitwise-table}
\end{table}

\subsubsection{Comparison trace}\label{subsubsec: olavm-comparison-trace}
OlaVM的comparison trace由7列组成。同时comparison操作的两个输入数会放入builtin的range check表。
\begin{itemize}
    \item \verb|clk|: 该comparison操作对应在主trace的时钟周期。
    \item \verb|sel|: 选择comparison的操作类型,sel=0为neq,sel=1为gte。
    \item \verb|op1 value|: bitwise操作的第一个输入值。
    \item \verb|op2 value|: bitwise操作的第二个输入值。
    \item \verb|flag|: bitwise操作的标志位状态,1是true,0是false。
    \item \verb|diff|: bitwise操作的两个输入的差,sel=1时为差的绝对值。
    \item \verb|diff_inv|: bitwise操作的两个输入的差的逆,差为0,则该值为0。
\end{itemize}

格式见下表 \ref{table: olavm-trace-comparison-table}。

\begin{table}[!ht]
    \centering \resizebox{\linewidth}{!}{
        \begin{tabular}{*{7}{|c}|}
            \hline
            clk & sel & op1 value & op2 value & flag & diff & diff\_inv   \\ \hline
        \end{tabular}}
    \caption{OlaVM的comparison的trace表结构}
    \label{table: olavm-trace-comparison-table}
\end{table}