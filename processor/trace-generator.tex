\subsection{Trace Generator}\label{subsec: : trace-generator}
当OlaVM在执行编译好的指令时,它会在每执行一条指令时,将对应的内存状态和寄存器状态保存在一张trace表中。
\subsubsection{Main Trace}\label{subsubsec: olavm-main-trace}
trac主表的列有定义如下表, 其中VM的时钟clk和取指令地址pc各占一列。寄存器状态从$\texttt{r}_0 - \texttt{r}_{15}$一共16列,内存占5列。trace的列定义见表\ref{table: olavm-trace-table},内存子表的定义见下节 \ref{subsubsec: olavm-trace-memory},
主表的clk字段是按照升序每行加1,可以唯一的标识该行:
\begin{table}[!ht]
    \centering \resizebox{0.6\linewidth}{!}{
        \begin{tabular}{*{8}{|c}|}
            \hline
            clk & pc & instruction & flag & r0 & $\cdots$ & r15 & memory  \\ \hline
        \end{tabular}}
    \caption{OlaVM的trace表结构}
    \label{table: olavm-trace-table}
\end{table}

\subsubsection{Memory Trace}\label{subsubsec: olavm-trace-memory}
OlaVM的内存trace由五列组成
\begin{itemize}
    \item \verb|addr|: 内存地址address的缩写。在表中是单调增的。
    \item \verb|clk|:VM的clock的缩写。操作内存时的VM时钟。在addr相同的情况下,clk是单调增的。
    \item \verb|pc|: VM操作内存时的指令地址。
    \item \verb|rw|: VM操作内存时的操作状态,目前状态有两种:0代表读和1代表写。
    \item \verb|value|: 内存当前存储的值。
\end{itemize}

格式见下表4 \ref{table: olavm-trace-memory-table}。

\begin{table}[!ht]
    \centering \resizebox{0.4\linewidth}{!}{
        \begin{tabular}{*{5}{|c}|}
            \hline
            addr & clk & pc & rw & value    \\ \hline
        \end{tabular}}
    \caption{OlaVM的trace表结构}
    \label{table: olavm-trace-memory-table}
\end{table}

下表5是一个内存子trace table的例子。按照后续约束规则,首先按照addr升序排列,同一个addr按照clk升序排列,如果该addr在未进行写操作之前进行读操作,读取值为默认0。
该表每行的addr和clk两个字段的组合可以唯一标识该行:

\begin{table}[!ht]
    \centering \resizebox{0.4\linewidth}{!}{
        \begin{tabular}{*{5}{|c}|}
            \hline
            addr & clk & pc & rw & value    \\ \hline
            100 & 4 & 3 & 1 & 0x25    \\ \hline
            100 & 7 & 5 & 0 & 0x25    \\ \hline
            100 & 11 & 7 & 0 & 0x25    \\ \hline
            100 & 15 & 3 & 1 & 0x74    \\ \hline
            100 & 17 & 6 & 0 & 0x74    \\ \hline
            102 & 8 & 6 & 1 & 0x11    \\ \hline
            102 & 13 & 9 & 0 & 0x11    \\ \hline
            102 & 19 & 19 & 1 & 0x53    \\ \hline
            102 & 20 & 20 & 0 & 0x53    \\ \hline
        \end{tabular}}
    \caption{OlaVM的trace表例子}
    \label{table: olavm-trace-memory-table-example}
\end{table}