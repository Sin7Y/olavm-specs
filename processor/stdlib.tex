\subsection{Stdlib}\label{subsec: : stdlib}
VM支持stdlib库的执行。比如无符号数的算术计算,常用hash函数(SHA256)等。
\subsubsection{u32 arithmatic}\label{subsec: u32-arithmatic}
OlaVM支持u32数据类型的算法计算。包括下边操作:
\begin{itemize}
    \item \verb|u32_add|
    \item \verb|u32_sub|
    \item \verb|u32_mul|
    \item \verb|u32_div|
\end{itemize}


\verb|u32 ADD|

u32的加法算法伪代码如下:

\begin{lstlisting}[language={}]
func u32_add(data_src_1: Felt, data_src_2: Felt) {
    MOV r_src1 data_src_1
    MOV r_src2 data_src_2

    range_check(r_src1, MAX_u32)
    range_check(r_src2, MAX_u32)

    ADD r_dst r_src1 r_src2
    %{
        (r_dst_hi, r_dst_lo) = split64(r_dst)
    %}

    MUL r_assert r_dst_hi 2^32
    ADD r_assert r_assert r_dst_low
    EQ r_dst r_assert

    return (r_dst_hi, r_dst_lo)
}
\end{lstlisting}


\verb|u32 SUB|

u32的减法算法伪代码如下:

\begin{lstlisting}[language={}]
func u32_sub(data_src_1: Felt, data_src_2: Felt) {
    MOV r_src1 data_src_1
    MOV r_src2 data_src_2

    range_check(r_src1, MAX_u32)
    range_check(r_src2, MAX_u32)

    local borrow_flag
    local difference
    %{
        if r_src1 > r_src2 {
            borrow_flag = 0
        } else {
            borrow_flag = 2^32
        }
        difference = borrow_flag + r_src1 - r_src2
    %}

    ADD r_assert1 r_src2 difference
    ADD r_assert2 r_src1 borrow_flag
    EQ r_assert1 r_assert2

    return (difference, borrow_flag)
}
\end{lstlisting}


\verb|u32 MUL|

u32的乘法算法伪代码如下:

\begin{lstlisting}[language={}]
func u32_mul(data_src_1: Felt, data_src_2: Felt) {
    MOV r_src1 data_src_1
    MOV r_src2 data_src_2

    range_check(r_src1, MAX_u32)
    range_check(r_src2, MAX_u32)

    MUL r_dst r_src1 r_src2
    %{
        (r_dst_hi, r_dst_lo) = split64(r_dst)
    %}

    MUL r_assert r_dst_hi 2^32
    ADD r_assert r_assert r_dst_low
    EQ r_dst r_assert

    return (r_dst_hi, r_dst_lo)
}
\end{lstlisting}


\verb|u32 DIV|

u32的除法算法伪代码如下:

\begin{lstlisting}[language={}]
func u32_div(data_src_1: Felt, data_src_2: Felt) {
    MOV r_src1 data_src_1
    MOV r_src2 data_src_2

    range_check(r_src1, MAX_u32)
    range_check(r_src2, MAX_u32)

    local quotient;
    local remainder;
    %{
        quotient = r_src1 / r_src2
        remainder = r_src1 % r_src2
    %}

    MUL r_assert r_src2 quotient
    ADD r_assert r_assert remainder
    EQ r_src1 r_assert

    return (r_dst_hi, r_dst_lo)
}
\end{lstlisting}
